\documentclass[12pt]{article}

% Автор: Илья Дудников
% Автор стиля: Сергей Копелиович

\usepackage{cmap}
\usepackage[T2A]{fontenc}
\usepackage[utf8]{inputenc}
\usepackage[russian]{babel}
\usepackage{graphicx}
\usepackage{amsthm,amsmath,amssymb}
\usepackage{listings}
\usepackage{color}
\usepackage{array}
\usepackage{epigraph}
\usepackage{multicol}
\usepackage{cancel}
\usepackage{float}
\usepackage{wrapfig}
\usepackage{caption}
\usepackage{subcaption}

\usepackage[usenames,dvipsnames]{xcolor}
\usepackage[russian,colorlinks=true,urlcolor=red,linkcolor=blue]{hyperref}
\usepackage{enumerate}
\usepackage{datetime}
\usepackage{fancyhdr}
\usepackage{lastpage}
\usepackage{verbatim}
\usepackage{tikz}
\usepackage{MnSymbol}
\usetikzlibrary{arrows,decorations.markings,decorations.pathmorphing}
\usepackage{pgfplots}
\usepackage{ifthen}
\usepackage{mathtools}
\usepackage{svg}
\usepackage{mathrsfs}

%\usepackage{tabls}
%\usepackage{tabularx}
%\usepackage{xifthen}
%\listfiles

\def\NAME{Ответы на билеты}
\def\SEASON{Алгоритмы и анализ сложности, ПИ, 3 семестр}

\sloppy
\voffset=-20mm
\textheight=235mm
\hoffset=-22mm
\textwidth=180mm
\headsep=12pt
\footskip=20pt

\parskip=0em
\parindent=0em

\setlength\epigraphwidth{.8\textwidth}

\newlength{\tmplen}
\newlength{\tmpwidth}
\newcounter{listcounter}

% Список с маленькими отступами
\newenvironment{MyList}[1][4pt]{
  \begin{enumerate}[1.]
  \setlength{\parskip}{0pt}
  \setlength{\itemsep}{#1}
}{       
  \end{enumerate}
}
% Вложенный список с маленькими отступами
\newenvironment{InnerMyList}[1][0pt]{
  \vspace*{-0.5em}
  \begin{enumerate}[(a)]
  \setlength{\parskip}{-0pt}
  \setlength{\itemsep}{#1}
}{       
  \end{enumerate}
  \vspace*{-0.5em}
}
% Список с маленькими отступами
\newenvironment{MyItemize}[1][4pt]{
  \begin{itemize}
  \setlength{\parskip}{0pt}
  \setlength{\itemsep}{#1}
}{       
  \end{itemize}
}

% Основные математические символы
\def\TODO{{\color{red}\bf TODO}}
\def\C{\mathbb{C}}       %
\def\Q{\mathbb{Q}}       %
\def\N{\mathbb{N}}       %
\def\R{\mathbb{R}}       %
\def\F2{\mathbb{F}_2}    %
\def\Z{\mathbb{Z}}       %
\def\INF{\t{+}\infty}    % +inf
\def\EPS{\varepsilon}    %
\def\EMPTY{\varnothing}  %
\def\PHI{\varphi}        %
\def\SO{\Rightarrow}     % =>
\def\EQ{\Leftrightarrow} % <=>
\def\t{\texttt}          % mono font
\def\c#1{{\rm\sc{#1}}}   % font for classes NP, SAT, etc
\def\O{\mathcal{O}}      %
\def\NO{\t{\#}}          % #
\def\XOR{\text{ {\raisebox{-2pt}{\ensuremath{\Hat{}}}} }}
\renewcommand{\le}{\leqslant}
\renewcommand{\ge}{\geqslant}
\newcommand{\q}[1]{\langle #1 \rangle}               % <x>
\newcommand\URL[1]{{\footnotesize{\url{#1}}}}        %
% \newcommand{\sfrac}[2]{{\scriptscriptstyle\frac{#1}{#2}}}  % Очень маленькая дробь
% \newcommand{\mfrac}[2]{{\scriptstyle\frac{#1}{#2}}}    % Небольшая дробь
\newcommand{\sfrac}[2]{{\scriptstyle\frac{#1}{#2}}}  % Очень маленькая дробь
\newcommand{\mfrac}[2]{{\textstyle\frac{#1}{#2}}}    % Небольшая дробь

\newcommand{\fix}[1]{{\color{fixcolor}{#1}}} % \underline
\def\bonus{\t{\red{(*)}}}
\def\ifbonus#1{\ifthenelse{\equal{#1}{}}{}{\bonus}}
\def\smallsquare{$\scalebox{0.5}{$\square$}$}

\newlength{\myItemLength}
\setlength{\myItemLength}{0.3em}
\def\ItemSymbol{\smallsquare}
\def\Item{\vspace*{\myItemLength}\ItemSymbol \ \ }

\newcommand{\LET}{%
  % [line width=0.6pt]
  \begin{tikzpicture}%
  \draw(0.8ex,0) -- (0.8ex,1.6ex);%
  \draw(0,1.6ex) -- (0.8ex,1.6ex);%
  \end{tikzpicture}%
  \hspace*{0.1em}%
}

% Отступы
\def\makeparindent{\hspace*{\parindent}\unskip}
\def\up{\vspace*{-0.5em}}%{\vspace*{-\baselineskip}}
\def\down{\vspace*{0.5em}}
\def\LINE{\vspace*{-1em}\noindent \underline{\hbox to 1\textwidth{{ } \hfil{ } \hfil{ } }}}
\def\BOX#1{\mbox{\fbox{\bf{#1}}}}
\def\Pagebreak{\pagebreak\vspace*{-1.5em}}

% Мелкий заголовок
\newcommand{\THEE}[1]{
  \vspace*{0.5em}
  \noindent{\bf \underline{#1}}%\hspace{0.5em}
  \vspace*{0.2em}
}
% Другой тип мелкого заголовка
\newcommand{\THE}[1]{
  \vspace*{0.5em} $\bullet$
  \noindent{\bf #1}%\hspace{0.5em}
  \vspace*{0.2em}
}

\newenvironment{MyTabbing}{
  \t\bgroup
  \vspace*{-\baselineskip}
  \begin{tabbing}
    aaaa\=aaaa\=aaaa\=aaaa\=aaaa\=aaaa\kill
}{
  \end{tabbing}
  \t\egroup
}

% Код с правильными отступами
\lstnewenvironment{code}{
  \lstset{}
%  \vspace*{-0.2em}
}%
{
%  \vspace*{-0.2em}
}
\lstnewenvironment{codep}{
  \lstset{language=python}
}%
{
}

% Формулы с правильными отступами
\newenvironment{smallformula}{
 
  \vspace*{-0.8em}
}{
  \vspace*{-1.2em}
  
}
\newenvironment{formula}{
 
  \vspace*{-0.4em}
}{
  \vspace*{-0.6em}
  
}

% Большая квадратная скобка
\makeatletter
\newenvironment{sqcases}{%
  \matrix@check\sqcases\env@sqcases
}{%
  \endarray\right.%
}
\def\env@sqcases{%
  \let\@ifnextchar\new@ifnextchar
  \left\lbrack
  \def\arraystretch{1.2}%
  \array{@{}l@{\quad}l@{}}%
}
\makeatother

% theorems
\makeatother
\usepackage{thmtools}
\usepackage[framemethod=TikZ]{mdframed}
\mdfsetup{skipabove=1em,skipbelow=0em}


\theoremstyle{definition}

\declaretheoremstyle[
    headfont=\bfseries\sffamily\color{ForestGreen!70!black}, bodyfont=\normalfont,
    mdframed={
        linewidth=2pt,
        rightline=false, topline=false, bottomline=false,
        linecolor=ForestGreen, backgroundcolor=ForestGreen!5,
    }
]{thmgreenbox}

\declaretheoremstyle[
    headfont=\bfseries\sffamily\color{NavyBlue!70!black}, bodyfont=\normalfont,
    mdframed={
        linewidth=2pt,
        rightline=false, topline=false, bottomline=false,
        linecolor=NavyBlue, backgroundcolor=NavyBlue!5,
    }
]{thmbluebox}

\declaretheoremstyle[
    headfont=\bfseries\sffamily\color{NavyBlue!70!black}, bodyfont=\normalfont,
    mdframed={
        linewidth=2pt,
        rightline=false, topline=false, bottomline=false,
        linecolor=NavyBlue
    }
]{thmblueline}

\declaretheoremstyle[
    headfont=\bfseries\sffamily\color{RawSienna!70!black}, bodyfont=\normalfont,
    mdframed={
        linewidth=2pt,
        rightline=false, topline=false, bottomline=false,
        linecolor=RawSienna, backgroundcolor=RawSienna!5,
    }
]{thmredbox}

\declaretheoremstyle[
    headfont=\bfseries\sffamily\color{RawSienna!70!black}, bodyfont=\normalfont,
    numbered=no,
    mdframed={
        linewidth=2pt,
        rightline=false, topline=false, bottomline=false,
        linecolor=RawSienna, backgroundcolor=RawSienna!1,
    },
    qed=\qedsymbol
]{thmproofbox}

\declaretheoremstyle[
    headfont=\bfseries\sffamily\color{NavyBlue!70!black}, bodyfont=\normalfont,
    numbered=no,
    mdframed={
        linewidth=2pt,
        rightline=false, topline=false, bottomline=false,
        linecolor=NavyBlue, backgroundcolor=NavyBlue!1,
    },
]{thmexplanationbox}

\declaretheorem[style=thmgreenbox, name=Определение]{Def}
\declaretheorem[style=thmbluebox, numbered=no, name=Пример]{Example}
\declaretheorem[style=thmredbox, name=Утверждение]{Prop}
\declaretheorem[style=thmredbox, name=Теорема]{Thm}
\declaretheorem[style=thmredbox, name=Свойство]{Property}
\declaretheorem[style=thmredbox, name=Лемма]{Lm}
\declaretheorem[style=thmredbox, numbered=no, name=Следствие]{Cons}

\declaretheorem[style=thmproofbox, name=Доказательство]{replacementproof}
\renewenvironment{proof}[1][\proofname]{\vspace{-10pt}\begin{replacementproof}}{\end{replacementproof}}


\declaretheorem[style=thmexplanationbox, name=Доказательство]{tmpexplanation}
\newenvironment{explanation}[1][]{\vspace{-10pt}\begin{tmpexplanation}}{\end{tmpexplanation}}

\declaretheorem[style=thmblueline, numbered=no, name=Замечание]{Rem}
\declaretheorem[style=thmblueline, numbered=no, name=Note]{note}
\declaretheorem[style=thmblueline, numbered=no, name=Обозначение]{notation}

% \newtheorem*{notation}{Notation}
\newtheorem*{Ex}{Упражнение}

% Определяем основные секции: \begin{Lm}, \begin{Thm}, \begin{Def}, \begin{Rem}
% \renewcommand{\qedsymbol}{$\blacksquare$}
% \theoremstyle{definition} % жирный заголовок, плоский текст
% \newtheorem{Thm}{\underline{Теорема}}[subsection] % нумерация будет "<номер subsection>.<номер теоремы>"
% \newtheorem{Lm}[Thm]{\underline{Lm}} % Нумерация такая же, как и у теорем
% \newtheorem{Ex}[Thm]{Упражнение} % Нумерация такая же, как и у теорем
% \newtheorem{Example}[Thm]{Пример} % Нумерация такая же, как и у теорем
% \newtheorem{Code}[Thm]{Код} % Нумерация такая же, как и у теорем
% \theoremstyle{plain} % жирный заголовок, курсивный текст
% \newtheorem{Def}[Thm]{Def} % Нумерация такая же, как и у теорем
% \theoremstyle{remark} % курсивный заголовок, плоский текст
% \newtheorem{Cons}[Thm]{Следствие} % Нумерация такая же, как и у теорем
% \newtheorem{Conj}[Thm]{Гипотеза} % Нумерация такая же, как и у теорем
% \newtheorem{Prop}[Thm]{Утверждение} % Нумерация такая же, как и у теорем
% \newtheorem{Rem}[Thm]{Замечание} % Нумерация такая же, как и у теорем
% \newtheorem{Remark}[Thm]{Замечание} % Нумерация такая же, как и у теорем
% \newtheorem{Algo}[Thm]{Алгоритм} % Нумерация такая же, как и у теорем

% Определяем ЗАГОЛОВКИ
\def\SectionName{unknown}
\def\AuthorName{unknown}

\newlength{\sectionvskip}
\setlength{\sectionvskip}{0.5em}
\newcommand{\Section}[4][]{
  % Заголовок
  \pagebreak
%  \ifthenelse{\isempty{#1}}{
    \refstepcounter{section}
%  }{}
  \vspace{0.5em}
%  \ifthenelse{\isempty{#1}}{
%    \addtocontents{toc}{\protect\addvspace{-5pt}}%
    \addcontentsline{toc}{section}{\arabic{section}. #2}
%  }{}
  \begin{center}
    {\Large \bf Раздел \NO{\arabic{section}}: #2} \\ 
    \vspace{\sectionvskip}
    \ifthenelse{\equal{#3}{}}{}{{\large #3}\\}
  \end{center}

  \LINE

  % Запомнили название и автора главы
  \gdef\SectionName{#2}
  \gdef\AuthorName{#4}

  % Заголовок страницы
  \lhead{\SEASON}
  \chead{}
  \rhead{\SectionName}
  \renewcommand{\headrulewidth}{0.4pt}

  \lfoot{Глава \NO{\arabic{section}}.}
  \cfoot{\thepage\t{/}\pageref*{LastPage}}
  \rfoot{Автор: \AuthorName}
  \renewcommand{\footrulewidth}{0.4pt}
}

\newcommand{\Subsection}[2][]{
  \refstepcounter{subsection}
  \vspace*{1em}
  \ifthenelse{\equal{#1}{}}
    {\addcontentsline{toc}{subsection}{\arabic{section}.\arabic{subsection}. #2}}
    {\addcontentsline{toc}{subsection}{\arabic{section}.\arabic{subsection}. \bonus\,#2}}
  {\color{blue}\bf\large \arabic{section}.\arabic{subsection}. \ifbonus{#1}\,{#2}} 
  \vspace*{0.5em}
  \makeparindent
}
\newcommand{\Subsubsection}[2][]{
  \refstepcounter{subsubsection}
  \vspace*{1em}
  \ifthenelse{\equal{#1}{}}
    {\addcontentsline{toc}{subsubsection}{\arabic{section}.\arabic{subsection}.\arabic{subsubsection}. #2}}
    {\addcontentsline{toc}{subsubsection}{\arabic{section}.\arabic{subsection}.\arabic{subsubsection}. \bonus\,#2}}
  {\color{blue}\bf\large \arabic{section}.\arabic{subsection}.\arabic{subsubsection}. \ifbonus{#1}\,#2}
  \vspace*{0.5em}
  \makeparindent
}

\makeatletter
\newcommand*{\encircled}[1]{\relax\ifmmode\mathpalette\@encircled@math{#1}\else\@encircled{#1}\fi}
\newcommand*{\@encircled@math}[2]{\@encircled{$\m@th#1#2$}}
\newcommand*{\@encircled}[1]{%
  \tikz[baseline,anchor=base]{\node[draw,circle,outer sep=0pt,inner sep=.2ex] {#1};}}
\makeatother

\newcommand{\Header}{
  \pagestyle{empty}
  \renewcommand{\dateseparator}{--}
  \begin{center}
    {\Large\bf 
     Алгоритмы и анализ сложности, 3 семестр ПИ,\\
    \vspace{0.3em}
    \NAME}\\
    \vspace{0.7em}
    {Собрано {\today} в {\currenttime}}
  \end{center}

  \LINE
  \vspace{0em}

  \renewcommand{\baselinestretch}{0.98}\normalsize
  \tableofcontents
  \renewcommand{\baselinestretch}{1.0}\normalsize
  \pagebreak
}

\newcommand{\BeginConspect}{
  \pagestyle{fancy}
  \setcounter{page}{1}
}

\definecolor{mygray}{rgb}{0.7,0.7,0.7}
\definecolor{ltgray}{rgb}{0.9,0.9,0.9}
\definecolor{fixcolor}{rgb}{0.7,0,0}
\definecolor{red2}{rgb}{0.7,0,0}
\definecolor{dkred}{rgb}{0.4,0,0}
\definecolor{dkblue}{rgb}{0,0,0.6}
\definecolor{dkgreen}{rgb}{0,0.6,0}
\definecolor{brown}{rgb}{0.5,0.5,0}

\newcommand{\green}[1]{{\color{green}{#1}}}
\newcommand{\black}[1]{{\color{black}{#1}}}
\newcommand{\red}[1]{{\color{red}{#1}}}
\newcommand{\dkred}[1]{{\color{dkred}{#1}}}
\newcommand{\blue}[1]{{\color{blue}{#1}}}
\newcommand{\dkgreen}[1]{{\color{dkgreen}{#1}}}

\newcommand{\Mod}[1]{\ (\mathrm{mod}\ #1)}

\DeclareMathOperator{\Real}{Re}
\DeclareMathOperator{\Imag}{Im}
\DeclareMathOperator{\lcm}{lcm}
\DeclareMathOperator{\sign}{sign}
\DeclareMathOperator{\Si}{Si}
\DeclareMathOperator{\const}{const}
\DeclareMathOperator{\Arg}{Arg}
\DeclareMathOperator{\Int}{Int}
\DeclareMathOperator{\Cl}{Cl}
\DeclareMathOperator{\diam}{diam}

\begin{document}
    \Header

    \BeginConspect

    \Section{? TODO: проблемы и доказательства}{}{Гагин Артур}
    
    \Subsection{Машины Тьюринга и тезис Чёрча}
   
    \begin{Def}[Машина Тьюринга]
        Да что за <<Машина Тьюринга>>?
        \begin{itemize}
            \item Абстрактная вычислительная машина.
            \item Формализация понятия алгоритма.
            \item Расширение конечного автомата.
            \item Лента (бесконечная).
            \item Головка записи-чтения (управляющее устройство), способная находиться в одном из множества состояний, которое конечно и точно задано.
            \item Это управляющее может перемещаться влево и вправо по ленте, читать и записывать в ячейки символы некоторого конечного алфавита.
            \item Существует $\epsilon$-символ, который заполняет все пустые клетки ленты.
            \item Управляющее устройство работает согласно правилам перехода, которые представляют алгоритм, реализуемый данной МТ. Каждое правило перехода предписывает машине, в зависимости от текущего состояния и наблюдаемого в текущей ячейке символа, записать в эту клетку новый символ, перейти в новое состояние и переместиться на одну клетку влево или вправо (существует некий <<синтаксический сахар>> --- остаться на месте). Некоторые состояния могут быть помечены как терминальные, и переход в любое из них означает конец работы, остановку алгоритма.
        \end{itemize}
        Это все очень интересно, но как насчет формализма?\\
        Формально: $M = \{Q, G, \epsilon, \Sigma, \delta, q_0, F\}$ --- запомните этот набор из семи элементов!
        \begin{itemize}
        \item $Q$ --- конечное, не являющееся пустым, множество состояний.
        \item $G$ --- конечный, не являющийся пустым, набор символов ленточного алфавита.
        \item $\epsilon$ --- единственный пустой символ.
        \item $\Sigma = G \backslash \{\epsilon\}$ --- набор входных символов.
        \item $\delta : (Q \backslash F) \times G \bcancel{\rightarrow}
        Q \times G \times \{L, R\}$ --- частичная функция, называемая функцией перехода, где $L$ --- сдвиг влево, а $R$ --- сдвиг вправо. Если $\delta$ не определена для текущего состояния и символа ленты, то машина останавливается.
        \item $q_0$ --- это начальное состояние.
        \item $F \subset {Q}$  --- набор конечных состояний.
        \end{itemize}
    \end{Def} 
    \begin{Rem}
        А что если мы хотим такое состояние, которое будет являться term при одном символе на ленте и nonterm при другом символе?\\
        Если немного подумать, то здесь все в порядке, поскольку мы можем сделать это состояние nonterm, перейти в другое term состояние, а при определенном символе сдвинуться, например, вправо, записав на ленту такой же символ, что был на ней.
    \end{Rem}
    \begin{Rem}
        Хоть любой конечный алфавит и не ограничен одними цифрами 0 и 1, очевидно, что мы всегда можем его представить в виде двоичных чисел, введя на нем порядок.
    \end{Rem}
    \begin{Def}[Частичная функция]
        частичная функция $f$ из множества $X$ в множество Y --- это функция из подмножества $S$ из $X$ (возможно, самого $X$) в $Y$ (обозначение: $\bcancel{\rightarrow}$). 
    \end{Def}
    \begin{Def}[Детерминированная и недетерминированная машины Тьюринга]
        Машина Тьюринга называется детерминированнй, если каждой паре состояни и ленточного символа соответствует не более одного правила. В ином случае, машина является недетерминированной.
    \end{Def}
    \begin{Def}[Тезис Черча/Тьюринга/Черча-Тьюринга]
        Есть ли отличие?\\
        На самом деле, все они говорят об одном, просто Черч в свое время придумал $\lambda$-исчисления, Тьюринг придумал Машину Тьюринга, а позже было показано, что эти формализмы эквивалентны.\\
        Сам тезис сформулируем следующим образом: Класс алгоритмически вычислимых частичных функций совпадает с классом всех функций, вычислимых на машине Тьюринга.
    \end{Def}
    \begin{Rem}
        Стоит понимать, что это именно тезис, а не теорема, ведь понятие <<алгоритмически вычислимая частичная функция>> неформально.
    \end{Rem}
    \begin{Def}[Вычислимая функция] Вычислимые функции — это множество функций         вида, $ f \colon N\ \to N $, которые могут быть реализованы на машине          Тьюринга.\\
        В качестве множества $N$ обычно рассматривается множество $B^{*}$ — множество слов в двоичном алфавите $B = \{0,1\}$, с оговоркой, что результатом вычисления может быть не только слово, но и специальное значение «неопределённость», соответствующее случаю, когда алгоритм «зависает». Таким образом, можно дать следующее определение $N$:\\
        $N=B^{*}\cup \{\operatorname{undef} \}$, где $B=\{0,1\}$, а $\operatorname{undef}$ — специальный элемент, означающий неопределённость.\\
        Роль множества $N$ может играть множество натуральных чисел, к которому добавлен элемент $ \operatorname{undef} $, и тогда вычислимые функции --- это некоторое подмножество натуральнозначных функций натурального аргумента. Удобно считать, что в качестве $N$ могут выступать различные счётные множества --- множество натуральных чисел, множество рациональных чисел, множество слов в каком-либо конечном алфавите и др. 
    \end{Def}
    \Subsection{Классы $RE$, $R$ и $co-RE$, доказательство $R = RE \cap co-RE$}
    \begin{Def}[Классы $RE$ и $co-RE$]
        Класс $RE$ (recursively enumerable) --- класс decision problems (проблемы принятия решения), для которых ответ <<да>> может быть проверен машиной Тьюринга за конечное время.
        \begin{itemize}
            \item Если на проблему ответ <<да>>, то существует некоторая процедура, которая требует конечного времени для определения этого.
            \item Ложь здесь отсутствует.
            \item Если на проблему ответ <<нет>>, то машина Тьюринга может остановиться, а может и не остановиться.
        \end{itemize}
        Класс $co-RE$ является дополнением к классу $RE$: ответ <<нет>> можно получить за конечное время абсолютно всегда, получение противоположного ответа может занять вечность.
    \end{Def}
    \begin{Def}[Формальный язык]
        Формальный язык (или просто язык) --- это множество конечным слов над конечным алфавитом.
    \end{Def}
    \begin{Def}[Класс $R$]
        $R$ --- класс decision problems, решаемых на МТ (набор всех рекурсивных языков). 
    \end{Def}
    \begin{Def}[Рекурсивный язык] Формальный язык является рекурсивным, если            существует полная машина Тьюринга (машина Тьюринга, которая останавливается для    каждого заданного ввода), которая, когда на вход подается конечная                  последовательность символов, принимает ее, если она принадлежит языку, и отвергает    ее в противном случае. 
    \end{Def}
    \begin{Thm}
        $R = RE \cap co-RE$.
    \end{Thm}
    \begin{proof}
        Обозначим за $X$ класс decision problems, содержащихся в классе $RE$, ответы <<да>> и <<нет>> на которые можно получить за конечное время. Очевидно, что $X \subset co-RE$ по определению, а также никакая другая задача, содержащаяся в $RE$ не содержится в $co-RE$. То есть, $X = RE \cap co-RE$. Однако $X = R$, так как это в точности класс decision problems, решаемых на машине Тьюринга, то есть: $R = X = RE \cap co-Re$.
    \end{proof}
    \Subsection{Проблемы Acceptence, Halting, Emptiness}
    \begin{Def}[Halting Problem]
        Проблема останова машины Тьюринга...
    \end{Def}

    \Section{? TODO: полностью}{}{Гагин Артур}

    \Section{?}{}{Гагин Артур}
        
    \Subsection{Построение универсальной машины Тьюринга}
    \begin{Def}[Универсальная машиина Тьюринга] Универсальная машина Тьюринга ---        такая машина, которая может заменить собой любую машину Тьюринга. Получив на     вход программу и входные данные, она вычисляет ответ, который вычислила бы по      входным данным машина Тьюринга, чья программа была дана на вход.
    \end{Def}
    \begin{Def}[Построение УМТ] 
    \end{Def}

    \Section{4?}{}{Гагин Артур}

    \Section{5?}{}{Гагин Артур}
    \Subsection{Определение классов сложности $P$, $NP$, $co-NP$}
    \begin{Def}[Класс сложности $P$] Классом $P$ называют множество задач, которые      могут быть решены за полиномиальное время в зависимости от размера входных данных.
    \end{Def}
    \begin{Def}[Класс сложности $NP$] Классом $NP$ называют множество задач,           решение которых возможно проверить на машине Тьюринга за время, не               превосходящее значения некоторого многочлена от размера входных данных (полином).
    \end{Def}
    \begin{Rem}
        Очевидно: $P \subset NP$ (если мы можем решить задачу за полиномиальное время, то мы можем проверить решение задачи, просто решив ее).
    \end{Rem}
\end{document}